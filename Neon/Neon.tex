\documentclass[]{article}
\usepackage{import}
\usepackage{/Users/kajsa-myblomdahl/Efimov/code/mypackages}
%opening
\title{Hyperradial calculations}
\author{}

\begin{document}

\maketitle

\section{Neon Potential}

\begin{table}[h!]
	\centering
	\footnotesize
	\begin{adjustwidth}{-0.1cm}{}
		\tabcolsep=0.10cm
		\begin{tabular}{||c c c c c c c c||} 
			\hline
			$n$ & $E^{\text{BO}}_{0n}$ & $E^{\text{adia}}_{0n}$ & $E^{\text{BO}}_{1n}$ & $E^{\text{adia}}_{1n}$ & $E^{\text{BO}}_{2n}$ & $E^{\text{adia}}_{2n}$ & $E^{\text{BO}}_{3n}$  \Tstrut\Bstrut \\ [0.7ex]
			\hline\hline  \Tstrut\Bstrut
			$0$   & $-$42.42  & $-$38.47 & $-$26.49 & $-$14.25 & $-$19.48 & & $-$14.86 \\
			$1$   & $-$30.77  & $-$23.43 & $-$20.60 & $-$12.79 & $-$13.42 & & \\
			$2$   & $-$24.78  & $-$16.01 & $-$14.51 & & & & \\
			$3$   & $-$18.08  & $-$12.96 & $-$13.05 & & & & \\
			$4$   & $-$14.29  & $-$12.44 & & & & & \\
			$5$   & $-$13.49  & & & & & & \\
			 [0.7ex]
			\hline 
		\end{tabular}
	\end{adjustwidth}
	\caption{Born--Oppenheimer energies $E^{\text{BO}}$ and adiabatic energies $E^{\text{adia}}$. The effective three-body potential was calculated with $N_{\theta}=N_{\phi} = 40$ and $N_{\rho}=215$. The ground state energy of the Ne dimer is  $-11.99 \, \text{cm}^{-1}$ and is taken as referense value for a bound three-body state.}
	\label{table:neon}
\end{table} 

\begin{table}[h!]
	\centering
	\footnotesize
	\begin{adjustwidth}{-0.1cm}{}
		\tabcolsep=0.10cm
		\begin{tabular}{||c c c c c c c c c c c c||} 
			\hline
			$\nu_{\text{max}}$ & 
			$E^{\nu_{\text{max}}}_{00}$ & $E^{\nu_{\text{max}}}_{01}$ & $E^{\nu_{\text{max}}}_{02}$ & $E^{\nu_{\text{max}}}_{03}$ &
			$E^{\nu_{\text{max}}}_{04}$ &
			$E^{\nu_{\text{max}}}_{10}$ &
			$E^{\nu_{\text{max}}}_{11}$ &
			$E^{\nu_{\text{max}}}_{12}$ &
			$E^{\nu_{\text{max}}}_{20}$ &
			$E^{\nu_{\text{max}}}_{21}$ & $E^{\nu_{\text{max}}}_{30}$ \Tstrut\Bstrut \\ [0.7ex]
			\hline\hline  \Tstrut\Bstrut
			$0$   & $-$38.47  & $-$23.43 & $-$16.01 & $-$12.96 & $-$12.44 & $\dots$ & $\dots$ & $\dots$ & $\dots$ & $\dots$ & $\dots$\\
			$1$   & $-$40.86  & $-$24.46 & $-$21.88 & $-$20.56 & $-$15.06 & $-$13.32 & $-$12.87 & $-$12.80 & $\dots$ & $\dots$ & $\dots$ \\
			$2$   & $-$41.08  & $-$25.29 & $-$24.33 & $-$22.92 & $-$19.58 & $-$14.74 & $-$13.87 & $-$13.04 & $-$12.92 & $-$12.32 & $\dots$ \\
			$3$   & $-$41.29  & $-$26.02 & $-$25.09 & $-$23.72 & $-$20.67 & $-$16.25 & $-$14.48 & $-$14.09 & $-$13.00 & $-$12.85 & $-$12.12 \\ [0.7ex]
			\hline 
		\end{tabular}
	\end{adjustwidth}
	\caption{Ground and excited state energies $E^{\nu_{\text{max}}}_{\nu n}$ calculated with an increasing number of channels.}
	\label{table:neon}
\end{table} 


The Neon potential used was the one developed by Aziz and Chen 1977. Referera till deras artikel

\begin{equation}
V(r) = \epsilon V^*(x)
\end{equation}

\begin{equation}
V^*(x) = A*\exp(-\alpha^*x+\beta^*x^2)-F(x)[c_6/x^6+c_8/x^8+c_{10}/x^{10}],
\end{equation}
where

\begin{equation}
F(x)=
\begin{cases}
\exp[-(D/x-1)^2], \quad &x<D\\
1, \quad &x \ge D
\end{cases}
\end{equation}




\end{document}
